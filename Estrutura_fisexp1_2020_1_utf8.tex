\chapter{Introdução}\vspace{-1.5cm}
A Apostila do curso de de Física Experimental apresenta 
%dos roteiros dos experimentos realizados no curso de Física Experimental I e de  
os conceitos b\'asicos relacionados com as análises de dados dos experimentos, bem como os métodos e instrumentos utilizados.
\vspace{-0.5cm}
\section*{Experimentos}

Ao longo do semestre realizaremos os seguintes experimentos, em  modo remoto:

\begin{descrip}
%\item [\bf INTRO] -- Introdução ao conceito de medidas -- Medições diretas e indiretas
\item[\bf EXP 1] -- Determinação do tempo de queda de uma moeda -- Tratamento estatístico dos dados
\item[\bf EXP 2] -- Medição do volume de uma moeda -- Propagação de incerteza
\item[\bf EXP 3] -- Movimento de um corpo em queda livre -- Aceleração da gravidade 
\item[\bf EXP 4] -- Sistema de partículas -- Colisões 
\end{descrip}



\vspace{-0.5cm}
\section*{Bibliografia}

O material completo da disciplina compreende a \href{https://fisexp1.if.ufrj.br/wp-content/uploads/2020/08/Apostila-1.pdf}{\textcolor {blue}
{Apostila de Conceitos Básicos de Física Experimental I}}, o  
\href{https://fisexp1.if.ufrj.br/wp-content/uploads/2020/08/GuiadoEstudante-2.pdf} {\textcolor {blue}
{Guia do Estudante}}
 e os textos complementares, todos disponí­veis no site \href{https://fisexp1.if.ufrj.br/}{\textcolor {blue}
{https://fisexp1.if.ufrj.br}}.


 
 Além disso, indicamos os seguintes livros para um estudo mais sólido dos conceitos básicos de análise de dados e da física dos fenômenos observados: Fundamentos da Teoria de Erros de José Henrique Vuolo~\cite{Vuolo},
 Curso de Fí­sica Básica - Mecânica de H. Moysés Nussenzveig~\cite{Moyses}
 e Fí­sica I, Mecânica, Sears \& Zemansky / Young \& Freedman~\cite{SearsZemansky}.