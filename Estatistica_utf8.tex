\chapter{Determinação da aceleração da gravidade (Queda livre) – Função Gaussiana}

\vspace{-0.7cm}

Neste experimento determinaremos o valor da aceleração da gravidade mediante o estudo da queda de uma bolinha de uma altura conhecida.  Como a determinação do tempo de queda da bolinha é uma grandeza com flutuações aleatórias vamos realizar uma análise estatística dos dados e estudar a função Gaussiana (Parte I).  Também vamos comparar o valor da gravidade determinado neste experimento com o encontrado no Experimento 4 e com o valor de referência da aceleração para a cidade do Rio de Janeiro (Parte II).

\section*{Processo experimental}
\begin{num}
\item Determine a altura $h$ de onde deixará cair a bolinha.
\item Faça 100 medidas do tempo de queda da esfera e registre-as na ordem cronológica. Pense em como serão feitas as medidas? Quem as fará e por quê? Como será o processo de registro das mesmas?
\end{num}

\section*{Análise de dados}


\subsection*{Parte I: Estatística e função Gaussiana}
\begin{num}
\item Calcular o valor médio, desvio padrão e a incerteza do valor médio (Capítulo~\ref{stat} da parte Conceitos Básicos) para: (a) as 20 últimas medidas, (b) 50 primeiras medidas e (c) para o conjunto completo de 100 medidas.
\item Compare os resultados obtidos no ponto anterior, colocando-os numa tabela e diga como variam estas três grandezas com o número de medidas.
\item Tome duas amostras aleatórias e independentes de 10 medições consecutivas cada uma. Calcule o valor médio, desvio padrão e incerteza do valor médio das amostras para ver como se relacionam entre sim e com o conjunto completo de medidas.
\item Usando os conhecimentos adquiridos compare o número de medições para o intervalo $[\bar{x}-\sigma;\bar{x}+\sigma]$ com o esperado. Analise seu resultado.
\item Calcule a partir da distribuição Gaussiana a probabilidade de obter valores mais desviados
que $\sigma$, 2$\sigma$, 3$\sigma$. Compare com o resultado obtido experimentalmente.
\item Realize um histograma de frequências relativas para seu conjunto de 100 medições da largura da mesa.
\item Calcule a distribuição de Gauss para o valor médio e o desvio padrão achados para o conjunto completo de dados e calcule quanto é a frequência relativa esperada para cada intervalo do histograma.
\item Superponha ao histograma realizado os valores achados no ponto anterior.
\item Que pode dizer do comportamento dos seus dados? Seguem uma distribuição Gaussiana? Justifique a sua resposta.
\end{num}

\subsection*{Parte II: Aceleração da gravidade}
\begin{num}
\setcounter{enumi}{9}

\item A partir dos dados de altura e tempo médio, determine o valor da aceleração da gravidade.
\item Compare este valor com o obtido no Experimento 4. 
\item Compare os dois resultados da aceleração da gravidade com o valor de referência para a cidade de Rio de Janeiro.  Discuta o resultado. O que pode dizer sobre as duas técnicas experimentais?

\end{num}
