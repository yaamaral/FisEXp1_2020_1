\chapter*{Estudo da conservação da energia -- Plano inclinado}

{\fontsize{76}{80}\usefont{T1}{pzc}{m}{n} \color{gray}
\begin{textblock*}{100mm}(.88\textwidth,-6.45cm)
4 \end{textblock*}
\begin{textblock*}{100mm}(.94\textwidth,-6.5cm)
+ \end{textblock*}

}
%\end{flushright}


\vspace{-0.5cm}

A partir dos dados do experimento 4 vamos estudar a conservação da energia mecânica para um carrinho que se movimenta num plano inclinado com atrito desprezível. Para isto, vamos utilizar os dados correspondentes ao movimento do carrinho no plano com maior inclinação.

\begin{num}

\item Determine a altura $h$ do carrinho para cada instante de tempo.
\item Determine a energia cinética ($K$), energia potencial ($U$) e a energia mecânica ($E$) para cada intervalo de tempo. Para facilitar a organização das informações, construa uma tabela.
\item Faça um gráfico que contenha a energia cinética, potencial e mecânica em função do tempo.
\item Discuta a partir do gráfico obtido, se há ou não conservação da energia mecânica. Justificar.
\item No caso da energia não se conservar, determine o ganho ou perda percentual.
\end{num}

{\bf Observação:}
\vspace{-0.5cm}
\begin{iten}
\item Para os cálculos de energia considere a aceleração da gravidade no Rio de Janeiro, sendo $g = (978,7 \pm 0,1)$ cm/s$^2$.  
\item Não esqueça colocar todos os cálculos de propagação de incerteza num Apêndice. 
\end{iten}


