\chapter*{Introdução}\vspace{-1.5cm}
Essa apostila consiste dos roteiros dos experimentos realizados no curso de Física Experimental I e de  conceitos b\'asicos relacionados com as análises de dados desses experimentos, bem como é com os métodos e instrumentos utilizados.
\vspace{-0.5cm}
\section*{Experimentos}

A longo do semestre realizaremos os seguintes experimentos:

\begin{descrip}
\item [\bf INTRO] -- Introdução ao conceito de medidas -- Medições diretas e indiretas
\item[\bf EXP 1] -- Medida do tempo de queda de uma esfera -- Tratamento estatístico dos dados
\item[\bf EXP 2] -- Medida do volume de um cilindro -- Propagação de incerteza
\item[\bf EXP 3] -- Movimento de um corpo em um plano inclinado -- Aceleração da gravidade 
\item[\bf EXP 4] -- Sistema de partículas -- Colisão elástica e inelástica
\item[\bf EXP 5] -- Movimento de um corpo rígido em um plano inclinado
\end{descrip}

\vspace{-0.5cm}
\section*{Bibliografia}

O material completo da disciplina compreende essa apostila, o {\bf Guia do Estudante} e os textos complementares, todos disponíveis no site \newline \url{https://fisexp1.if.ufrj.br}. Além disso, indicamos os livros abaixo para um estudo mais sólido dos conceitos básicos de análise de dados e da física dos fenômenos observados.
\vspace{-0.5cm}
\begin{descrip}
\item Fundamentos da Teoria de Erros – José Henrique Vuolo – Editora Edgar Blücher Ltda. 

\item Curso de Física Básica 1 – Mecânica, H. Moysés Nussenzveig – Ed. Edgard Blücher Ltda.

\item Física I – Mecânica, Sears \& Zemansky / Young \& Freedman – 12a. Edição, Pearson.
\end{descrip}