% !TEX encoding = IsoLatin9
\chapter*{Introdu\c c\~ao}\vspace{-1.5cm}
Essa apostila consiste dos roteiros para os experimentos realizados no curso de F\'\i sica Experimental I da UFRJ. \vspace{-0.5cm}
\section*{Experimentos}

Ao longo do semestre realizamos os seguintes experimentos:

\begin{descrip}
\item [\bf INTRO] -- Introdu��o ao conceito de medidas -- Medi��es diretas e indiretas
\item[\bf EXP 1] -- Medida do tempo de queda de uma esfera -- Tratamento estat��stico dos dados
\item[\bf EXP 2] -- Medida do volume de um cilindro -- Propaga��o de incerteza
\item[\bf EXP 3] -- Movimento de um corpo em um plano inclinado -- Acelera��o da gravidade 
\item[\bf EXP 4] -- Sistema de part��culas -- Colis�o el�stica e inel�stica
\item[\bf EXP 5] -- Movimento de um corpo r��gido em um plano inclinado
\end{descrip}

\vspace{-0.5cm}
\section*{Bibliografia}

O material completo da disciplina compreende essa apostila, a apostila de conceitos b�sicos de F�sica Experimental I, o {\bf Guia do Estudante} e os textos complementares, todos dispon��veis no site \newline \url{https://fisexp1.if.ufrj.br}. Al�m disso, indicamos os livros abaixo para um estudo mais s�lido dos conceitos b�sicos de an�lise de dados e da f�sica dos fen�menos observados.
\vspace{-0.5cm}
\begin{descrip}
\item Fundamentos da Teoria de Erros - Jos� Henrique Vuolo - Editora Edgar Bl�cher Ltda. 

\item Curso de F��sica B�sica 1 - Mec�nica, H. Moys�s Nussenzveig -  Ed. Edgard Bl�cher Ltda.

\item F��sica I ; Mec�nica, Sears \& Zemansky / Young \& Freedman - 12a. Edi��o, Pearson.
\end{descrip}