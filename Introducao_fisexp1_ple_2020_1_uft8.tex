% !TEX encoding = IsoLatin9
\chapter*{Introdu\c c\~ao}\vspace{-1.5cm}
Esta apostila apresenta os conceitos b�sicos relacionados com as an�lises de dados dos experimentos,   bem como, os m�todos e instrumentos utilizados 

\section*{Experimentos}

Ao longo do semestre realizamos os seguintes experimentos, em modo remoto:

\begin{descrip}
\item [\bf INTRO] -- Introdu��o ao conceito de medidas -- Medi��es diretas e indiretas
\item[\bf EXP 1] -- Determina��o do tempo de queda de uma moeda -- Tratamento estat��stico dos dados
\item[\bf EXP 2] -- Medi��o do volume de uma moeda -- Propaga��o de incerteza
\item[\bf EXP 3] -- Movimento de um corpo em queda livre -- Acelera��o da gravidade 
\item[\bf EXP 4] -- Sistema de part��culas -- Colis�es
\end{descrip}

\vspace{-0.5cm}
\section*{Bibliografia}

O material completo da disciplina compreende essa apostila, a apostila de conceitos b�sicos de F�sica Experimental I, o {\bf Guia do Estudante} e os textos complementares, todos dispon��veis no site \newline \url{https://fisexp1.if.ufrj.br}. Al�m disso, indicamos os seguintes livros para um estudo mais s�lido dos conceitos b�sicos de an�lise de dados e da f�sica dos fen�menos observados: Fundamentos da Teoria de Erros de Jos� Henrique Vuolo~\cite{Vuolo}, Curso de F��sica B�sica - Mec�nica de H. Moys�s Nussenzveig~\cite{Moyses} e F��sica I (Mec�nica) de Sears \& Zemansky / Young \& Freedman~\cite{SearsZemansky}.