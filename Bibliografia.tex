%\chapter{Referências}
%\chapter{Bibliografia}

%\label{sec:referencias_bibliograficas}
\vspace{-0.7cm}

\begin{thebibliography}{9}
\bibitem{Vuolo} Fundamentos da Teoria de Erros,  José Henrique Vuolo,  $2.^a$ Edição 1996, Editora Edgar Blücher Ltda,

\bibitem{Moyses} Curso de Física Básica,  – Mecânica Volume 1), H. Moysés Nussenzveig, $2.^a$ Edição 2015, Ed. Edgard Blücher Ltda,

\bibitem{SearsZemansky} Física I – Mecânica, Sears \& Zemansky / Young \& Freedman, $14.^a$ Edição 2016, Pearson Education do Brasdil Ltda., 

\bibitem{Galante2020}
{\it Two-penny physics: Teaching 2D linear momentum conservation}, 
L. Galante e I. Gnesi, 
American Journal of Physics {\bf88}, 279  (2020).

\bibitem{deJesus2016}
{\it Uma visão diferenciada sobre o ensino de forças impulsivas
usando um smartphone}, 
V. L. B. de Jesus e D. G. G. Sasaki, 
Revista Brasileira de Ensino de Física {\bf38}, 1303  (2016).
\end{thebibliography}

