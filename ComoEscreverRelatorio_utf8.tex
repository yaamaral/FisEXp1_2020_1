\chapter{Como escrever um relatório?}
\label{sec:relatorios}
\vspace{-0.7cm}

A idéia desta nota é dar aos alunos de Física Experimental I algumas dicas e re\-co\-men\-da\-ções de como escrever um relatório. Infelizmente, não existe uma “receita” para isto, pois há várias maneiras de fazer um relatório, dependendo do tipo de trabalho realizado e de quem o escreva. Portanto, a organização do relatório pode ser diferente apresentando diferentes distribuições de seções. Nesta nota propõe-se uma estrutura básica com algumas sugestões, mas será com a experiência, com a prática e com as sucessivas correções do professor que os alunos aprenderão a fazê-lo. Escrever um relatório é um aprendizado que se obtém aos poucos.
 
O ponto principal a ser tido em conta é que no relatório deve-se apresentar os resultados obtidos de forma clara e concisa. Para isto, deve-se expor cuidadosamente quais são os objetivos do trabalho realizado, os conceitos físicos básicos necessários para a realização do experimento e como ele foi realizado, entre outros. O relatório tem que ser escrito de modo que um leitor que nunca tenha realizado o experimento descrito, ou a pesquisa realizada, seja capaz de entender e até reproduzir o trabalho a partir do conhecimento adquirido na sua leitura. Para começar, sugere-se a seguinte distribuição:

\begin{itemize}
\item {\bf Título e autores:}  O título deve descrever claramente o conteúdo do trabalho. O relatório tem que ter o(s) nome(s) do(s) autor(es) e as informações relevantes referentes a ele(s). 

\item {\bf Resumo:} Deve dar uma visão completa do trabalho realizado. De forma breve, deve-se descrever qual é o objetivo do mesmo, o que foi feito e qual foi o resultado obtido. 

\item {\bf Introdução:} 
Nela expõem-se as motivações do trabalho e os objetivos a serem atingidos. Deve-se apresentar uma revisão da informação existente sobre o tema em questão. Também, deve-se incluir uma explicação teórica mínima (não copiada de livro, mas elaborada pelos alunos) que permita a compreensão do trabalho e como esta informação está aplicada ao experimento específico. 

\item {\bf Método experimental ou Descrição do experimento:} 
Deve-se descrever em detalhe a configuração experimental utilizada, os métodos utilizados para a realização das medições, incluindo a fundamentação física. Deve-se realizar uma descrição dos aspectos relevantes dos dispositivos e equipamentos utilizados, especificando suas características importantes (precisão dos instrumentos, intervalos de medição, etc). Pode-se representar esquematicamente o dispositivo empregado para a realização do experimento de forma a acompanhar as explicações e facilitar a compreensão do leitor. 

\item {\bf Resultados e discussão: }
Esta seção tem que ser uma continuação natural da In\-tro\-du\-ção e do Método experimental ou Descrição do experimento. Deve-se incluir tabelas dos dados colhidos junto com as suas incertezas e a explicação de como foram avaliadas essas incertezas. Também deve ser realizada uma descrição de como a análise de dados foi realizada e como os resultados foram obtidos. Deve-se incluir também gráficos, junto com as curvas de ajuste dos dados realizados.  Além da análise dos dados, é fundamental realizar uma discussão dos mesmos: sua validade, precisão e a sua interpretação. Dependendo do caso, pode-se realizar uma proposição de um modelo para a descrição dos resultados ou realizar uma comparação com o modelo teórico já discutido na introdução. Caso seja necessária a utilização de equações, elas devem estar explicitadas ou, se já foram introduzidas anteriormente (na introdução), através de uma referência ao número de equação correspondente. 

Levar em conta que, dependendo do relatório e do trabalho apresentados, pode-se separar esta seção em duas independentes, uma de resultados e outra de discussões. 

Figuras e tabelas: cada figura ou tabela deve estar numerada e deve conter uma legenda ao pé que permita entendê-la. A descrição detalhada da figura deve estar incluída também no texto e referenciada pelo número. Os gráficos são considerados figuras, então deverão ser numerados de forma correlacionada com as mesmas.

\item {\bf Conclusões:} 
Deve conter uma discussão de como a partir dos resultados obtidos mostra-se que as hipóteses e objetivos do trabalho foram satisfeitos ou não. Espera-se que a discussão do trabalho seja feita de forma crítica podendo-se propor melhoras ao trabalho realizado, tanto na metodologia empregada quanto nas propostas para ampliar o objetivo do experimento no futuro.

\item {\bf Referências:} 
Deve-se informar a bibliografia citada durante o desenvolvimento do trabalho. A bibliografia pode estar relacionada ao modelo teórico discutido, a referências de equipamento utilizado, ou a artigos de referência no qual o trabalho foi baseado.

\item {\bf Apêndice:} 
Caso seja necessário, pode-se anexar um ou mais apêndices com in\-for\-ma\-ção complementar que ajude a esclarecer o conteúdo das partes anteriores (cálculos realizados para obter um dado resultado, estimativa de incertezas, etc.), mas que no corpo principal do relatório desviariam a atenção do leitor. No(s) a\-pên\-di\-ce(s) coloca-se geralmente informação adicional necessária, mas não fundamental.

\end{itemize}