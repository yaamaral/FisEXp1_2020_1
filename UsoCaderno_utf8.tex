\chapter{Caderno de laboratório}

\vspace{-0.7cm}

\begin{enumerate}
\item {\bf É um documento.} Nele se tem todos os registros cronológicos de um experimentos ou ideia. Portanto, deve ter datas, sem rasuras nem espaços em branco, sem inserções e se possível assinado por quem realizou as anotações.

\item {\bf É pessoal}. Pode haver outros cadernos de uso compartilhado, por exemplo, para equipamentos ou instrumentos de laboratório, etc., onde se registram informações de uso geral, como mudanças introduzidas em configurações experimentais ou estado de conservação dos equipamentos. Mas o caderno de laboratório contem ideias, propostas e modo de colocar a informação que são pessoais, próprias de cada pessoa.

\item {\bf É um registro de anotação em sequência.} Não se devem intercalar resultados nem se corrigir o que está escrito. Em caso de se detectar um erro, se anota na margem o erro encontrado e a página na qual se corrige. Isto permite saber se o erro pode-se voltar a encontrar e a partir de que dados foi corrigido. Por este mesmo motivo não se deve escrever a lápis.

\item {As páginas devem ser numeradas.} Isto permite fazer referência de forma fácil e organizada às anotações anteriores, assim como também indicar na margem onde se corrigem os erros.

\item {As fórmulas e figuras devem ter uma numeração consistente e interna.} Um exemplo prático é numerar todas as fórmulas dentro de cada página ou folha e citá-las por página–fórmula. É importante numerar todas as fórmulas, pois não sabemos no futuro qual necessitaremos citar ou utilizar.

\item {Referências completas.} No caso em que se deva utilizar uma referência externa (roteiro do experimento, artigo, livro, etc.), esta referência deve ser completa. Se uma referência é citada com frequência pode-se utilizar a última página do caderno para registrá-la e citá-la por seu número. Quando citamos alguma coisa, sempre acreditamos que vamos nos lembrar de onde saiu, mas isto só é assim a curto prazo.

\item {\bf Deve-se escrever todos os resultados.} Indicar sempre a maior quantidade de in\-for\-ma\-ção possível do experimento. Todas as condições experimentais devem ser corretamente registradas e deve-se utilizar diagramas claros das configurações experimentais e indicando também cada vez que há uma mudança. Um dado ou informação que hoje parece irrelevante em função do nosso modelo da realidade, pode resultar vital ao descobrir que nossas ideias estavam erradas ou eram incompletas. A falta de um dado de aparência menor pode invalidar tudo o que foi  realizado.

\item {\bf Deve-se escrever o plano.} O que é que se pretende medir, o que é que se procura e as considerações ou razões pelas quais se faz o experimento. O planejamento do experimento e as ideias a serem realizadas devem ser explícitas. A anotação sequencial permite seguir a evolução das idéias, dado vital também para interpretar os resultados, pois os preconceitos condicionam o que se mede e como se mede. Saber o que se pensava no momento de medir vai nos indicar se nesse momento tivemos uma determinada precaução que depois demostrou ser fundamental.

\item {\bf Deve-se escrever as conclusões.} O mesmo vale para o planejamento do experimento.

\item {\bf Fazer uma reorganização periódica das ideias.} Se uma ideia tem evoluído desde o inicio do experimento, é conveniente periodicamente fazer um quadro da situação, passando a limpo o que foi feito, para não ter que reconstruir a história a cada vez.

\end{enumerate}