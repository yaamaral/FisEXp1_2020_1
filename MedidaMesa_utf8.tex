\chapter{Medida da largura de uma mesa}

\vspace{-0.7cm}


Neste experimento determinaremos a largura de uma mesa de duas formas diferentes: (1) utilizando uma régua de 1 m de comprimento e (2) utilizando uma régua de 15 cm de comprimento. A partir destas medições vamos estudar os conceitos de medidas diretas e flutuações aleatórias.

\vspace{-0.7cm}
\section*{Procedimento experimental:}

\begin{num}
\item Determine a largura de uma mesa utilizando uma régua de pelo menos 1 m de comprimento.  Anote no seu caderno de laboratório o resultado obtido. 
\item Realize medidas da largura da mesma mesa utilizando uma régua de 15 cm de comprimento. Cada integrante da dupla deverá realizar 30 medidas diferentes. Registre no seu caderno de laboratório os valores medidos separadamente para cada integrante na ordem cronológica, na forma de uma tabela (Tabela~1).  Esta tabela será complementada na análise de dados.
\end{num}

\vspace{-0.7cm}
\section*{Análise de dados:}

\begin{num}
\item Para cada conjunto de 30 medições calcule o valor médio, o desvio padrão e a incerteza do valor médio (Capítulo~\ref{stat} da parte Conceitos Básicos). Para auxiliar nos cálculos, complete a Tabela 1 da seguinte forma:

\begin{table}[h]
\begin{center}

\hspace{-0.8cm}
  \begin{tabular}[m]{ l | c | c | c | c |}
    \cline{2-5}	
    &  ~~~$i$~~~ & ~~~$x_i$~~~ & $\delta_i = x_i - \bar{x}$ & $\delta_i^2 = (x_i - \bar{x})^2$ \\\cline{2-5}
   & & & & \\
    &  & & &  \\ \cline{2-5}
    Somas & & & & \\ \cline{2-5}
  \end{tabular}
  \caption*{Tabela 1: Tabela auxiliar para a análise do conjunto de 30 medidas.}
\end{center}
\vspace{0cm}
\end{table}

\item Para um dos dois conjuntos de dados construa um histograma (Seção~\ref{histo} da parte Conceitos Básicos) em uma folha de papel milimetrado. Lembre que o número de barras depende do conjunto de dados e o número total de medições. Neste caso particular, o número aconselhável de barras está entre 5 e 8. Se possível, escolha os valores míınimo e máximo do eixo horizontal de forma que o histograma do segundo conjunto de dados possa ser adicionado ao mesmo gráfico (ver item 6).

\item Marque no histograma a posição do valor médio achado no item anterior. Como se distribuem as medições ao redor do valor médio?

\item Analise como se comparam os valores médios encontrados com o valor de referência, ou seja, a medição realizada com a régua grande.

\item Caso existam incertezas sistemáticas, discuta sobre elas e como poderia evitá-las refazendo as medições.

\item Superponha o histograma do segundo conjunto de dados ao primeiro, distinguindo claramente as medições de um com as do outro (traços de distinta cor, por exemplo). Caso não seja possível usar a mesma figura, utilize uma nova folha de papel milimetrado.

\item Comparando os valores médios e desvios padrão obtidos pelos dois integrantes e olhando as duas distribuições, é possível unificar todas as medidas representadas nos dois conjuntos num único histograma? Existem diferenças entre a forma de medir dos dois membros do grupo? Discuta e justifique a sua resposta.

\end{num}


\vspace{-0.7cm}
\section*{Opcional} 

Realize novamente as suas medidas tendo em conta os cuidados discutidos para eliminar suas incertezas sistemáticas. Para este conjunto de 30 medições, calcule o valor médio, o desvio padrão e a incerteza do valor médio e compare com o valor de referência. Conseguiu eliminar as incertezas sistemáticas?  Discuta. 
