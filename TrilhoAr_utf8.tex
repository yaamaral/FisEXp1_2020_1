\chapter{Trilho de Ar}\label{trilho}
\vspace{-0.3cm}
O trilho de ar é um sistema que permite estudar movimentos unidimensionais reduzindo drasticamente as forças de atrito habitualmente presentes. Ele é composto de chapas metálicas de perfil reto, com pequenos orifícios regularmente espaçados nas faces  laterais.  

Injeta-se ar comprimido dentro do trilho que sai através dos orifícios gerando desta forma um colchão de ar entre o trilho e o carrinho de cerca de 0,5 mm de espessura. Este colchão de ar faz com que o carrinho "flutue", provocando assim uma grande redução do atrito.  O atrito residual é devido principalmente à fricção com o ar. Nas extremidades do trilho sempre deve-se colocar os pára-choques formado por um suporte com elásticos. 

O sistema trilho de ar e carrinho devem ser cuidadosamente tratados para evitar que eles se sofram deformações ou marcas que comprometam a redução do atrito.  Para isto devemos evitar choques mecânicos fortes, tanto ente o carrinho e o trilho como entre dois carrinhos. Evitar quedas dos carrinhos, mesmo que sejam de uns poucos centímetros, manuseando-os com segurança e muito cuidado. Em continuação enumeramos alguns cuidados extras na hora de utilizar o sistema trilho-carrinho, a saber:
\vspace{-0.3cm}
\begin{iten}

\item Nunca movimente os carrinhos sobre o trilho quando não houver ar saindo pelos orifícios do trilho ou se o ar que sai é muito fraco (neste caso deve se aumentar a potência do ar comprimido), pois serão produzidos arranhões.

\item Quando se colocar massas em cima dos carros, é fundamental que estas sejam distribuídas simetricamente para evitar desbalanceamentos do carrinho quando flutua podendo encostar certas partes dele no trilho.  Desta forma não só poderão ser produzidos arranhões no trilho senão como a hipóteses de atrito desprezível não será verificada. 

\item O trilho é apoiado sobre pequenas hastes numa base em perfil de alumínio. Estas hastes têm como função permitir o nivelamento do trilho. Com o tempo e o uso constante, o trilho de ar tende a se deformar criando "barrigas".  A consequência principal destas barrigas é os carrinhos passam a ter um movimento irregular. O nivelamento do trilho é uma operação trabalhosa e delicada, por isso deve-se estar seguro da necessidade de nivelamento antes de começar a mexer.

\end{iten}