
\chapter{Sistema de partículas -- Colisões}
\label{chap:colisao}
\vspace{-0.7cm}



\section{Introdução}
Neste experimento estudaremos colisões entre dois corpos. Em particular, procuraremos verificar experimentalmente se a conservação de momento linear total do sistema está presente. Também iremos ana-lisar uma possível conservação de energia mecânica total, explorando a diferença entre colisões elásticas e inelásticas. Na tomada de dados o movimento é gravado em um filme. A variação das posições dos corpos é analisada posteriormente. Programas de computador como aquele já utilizado no experimento anterior, o Tracker 
(Apêndice~\ref{sec:tracker}), são bastate úteis nessa análise. A análise do movimento com aplicativos no próprio celular é também possível.

A realização deste experimento em um curso remoto merece alguns comentários, dadas suas peculiaridades em relação à presença de atrito. Na versão presencial da disciplina de Física Experimental 1 da UFRJ, os alunos trabalham em sala com uma montagem usada para minimizar o atrito de dois carrinhos com a base: o trilho de ar. Em nosso curso remoto, os alunos não tem à disposição em casa uma ferramenta eficiente como o trilho de ar para minimizar o atrito. Assim, dividimos o experimento em duas partes. Na primeira parte o estudante faz a análise de dados experimentais nos moldes do curso presencial, porém utilizando dados obtidos previamente no Laboratório de Física Experimental 1. Na segunda parte (de realização opcional) o estudante faz a tomada de dados em casa, fazendo colidir objetos aos quais ele tem acesso fácil, como moedas. Este é um desafio final na disciplina: estudar colisões fora das condições controladas no laboratório didático, planejar e realizar sua montagem experimental, e analisar um sistema físico no qual as forças de atrito provavelmente serão observáveis. Para essa segunda parte, apontamos ainda uma versão simplificada do experimento que, fazendo suposiões sobre o atrito, permite fazer um teste da conservação de momento linear em uma colisão, mesmo sem o uso de celular ou computador \cite{Galante2020}. \footnote{{\it Two-penny physics: Teaching 2D linear momentum conservation}, Lorenzo Galante e Ivan Gnesi, American Journal of Physics {\bf88}, 279 (2020).} Outros exemplos de experimentos envolvendo colisões que os alunos podem fazer em casa estão na página da disciplina  \href{https://fisexp1.if.ufrj.br}{\textcolor {blue} {Física Experimental 1}} , no site do IF . 


\section{Colisão unidimensional sem atrito}

Neste experimento estudaremos as colisões e seu caráter elástico ou inelástico. Analisaremos as conservações de momento linear e energia mecânica de um sistema unidimensional de dois carrinhos que colidem entre si em um trilho de ar com atrito desprezível. Será utilizada uma gravação de um filme. Iremos analisá-la com o programa Tracker (Apêndice~\ref{sec:tracker}) para computador ou VidAnalysis (Apêndice~\ref{sec:vidanalysis}) para celular para levantamento de dados. O aluno também poderá usar outros programas ou aplicativos que permitam atingir os mesmos objetivos.

Pense sobre o planejamento desse experimento. Quais grandezas devem ser medidas diretamente para que seja possível avaliar as conservações de momento linear e energia mecânica? 

Siga o roteiro e as orientações do professor para fazer o experimento. Faça todas as anotações que julgar serem necessárias, elas serão importantes quando você for analisar os dados. Ao preparar o relatório,  tome como base as orientações do Apêndice ~\ref{sec:relatorios}  da Apostila do curso e as anotações realizadas durante o experimento. As discussões contidas no roteiro abaixo serão importantes para a elaboração do seu relatório.



Reflita sobre as seguintes questões e sugestões:

\begin{enumerate}
\item Qual é o objetivo desse experimento?
\item O que é um processo de colisão?
\item O filme mostra o movimento de dois carrinhos que deslizam sobre um trilho horizontal; há uma camada de ar entre o trilho e a base dos carrinhos, a fim de minimizar o atrito. Que tipo de movimento sobre o trilho é esperado para cada carrinho  antes e após a colisão? Pense nas forças que atuam sobre cada um deles.
\item  Considere a situação onde dois carrinhos colidem entre si ao se movimentarem sobre um trilho de ar horizontal com atrito desprezível; espera-se que tanto o momento linear como a energia mecânica se conservem nas colisões? O que define a diferença entre as colisões elástica e inelástica? Desenvolva as expressões matemáticas para conservação de momento linear e energia mecânica deste sistema unidimensional para os dois tipos de colisão, em termos das grandezas medidas no experimento. 
\item Como verificar experimentalmente se o momento linear do centro de massa do sistema é conservado?
\end{enumerate}

\vspace{-0.3cm}
\section{Procedimento Experimental e Levantamento de Dados}
Você terá acesso ao filme elasti.mp4 que deverá ser aberto com o programa Tracker ou VidAnalysis.
É interessante você observar esse filme: um carrinho (incidente), que chamaremos de $A$, move-se em direção a outro carrinho, denominado $B$, que está inicialmente em repouso. Ocorre o choque, mediado por ``para-choques''  feitos com elásticos esticados. Em seguida, $B$ passa a se mover, enquanto que $A$ continua a se mover, porém mais lentamente do que antes do choque. As massas dos carrinhos são $m_{A}=287,9 \pm 0,2$~g e $m_{B}=179,4 \pm 0,2$~g.
O comprimento total do trilho é $200$ cm; esse valor é importante para calibrar os comprimentos (escala da sua filmagem). Essa informação será usada quando for atribuir um valor à barra de medição do Tracker. Quando for estimar seu erro na calibração, note que há uma deformação da geometria do trilho no filme.

\begin{enumerate}
\item Proceda fazendo a tomada de dados da posição dos dois carrinhos antes e depois do choque. Precisaremos, para cada carrinho, aproximadamente 10 pontos antes e 10 pontos depois do choque. Entretanto, para minimizar o erro nas velocidades determinadas, é aconselhável que esses pontos não sejam “instantes sucessivos" registrados pelo Tracker ou VidAnalysis. 
Você pode escolher um intervalo de $0,1$ s no seu registro de dados (há vários ``frames''  do Tracker
ou VidAnalysis entre eles). 
\item  Analise o movimento de cada carrinho (tomada de dados) separadamente. Você terá que escolher um ponto de referência em cada carrinho para acompanhá-lo “manualmente" no programa.
Então terá que estimar a incerteza da sua medida de posição: amplie a imagem
e estime com que precisão consegue identificar o ponto de referência. Por exemplo, para uma ampliação de $200$ vezes, a incerteza é da ordem de $3$ mm.
\item Lembre-se de escolher um único sistema de referência para a determinação da posição em função do tempo. 
\item Construa uma tabela da posição de cada carrinho em função do tempo. 
\end{enumerate}

\section{Análise de dados e discussão dos resultados}
\begin{enumerate}

\item O instante de colisão pode ser obtido diretamente a partir da tabela dos dados?  Faça um gráfico da posição em função do tempo para os dois carrinhos e determine o instante em que eles colidem. 
\item Determine as velocidades dos carrinhos antes e depois da colisão a partir do ajuste linear dos dados. Alternativamente, você pode usar o método gráfico para tal determinação. As duas abordagens estão descritas em apêndices na apostila do curso.
\item Analise o comportamento do momento linear e da energia mecânica do sistema antes e depois do choque. Houve conservação dessas grandezas? Que conclusões você pode tirar desses resultados?
%\item Faça uma previsão dos valores das velocidades finais que você espera em função do modelo teórico.
%\item Compare os valores para as velocidades finais obtidas experimentalmente com as previstas no ponto anterior.  Discuta.

\item Calcule a porcentagem de perda de energia cinética, dada por:
\[
\frac{| K_f - K_i|}{K_i}
\]
\noindent
onde $K_i$ e $K_f$ são a energia cinética inicial e final, respectivamente. Discuta os resultados obtidos.
\end{enumerate}


%\item Calcule a posição do centro de massa em função do tempo (adicione as colunas correspondentes na tabela de posição dos carrinhos em função do tempo) e adicione estes pontos ao gráfico de posição em função do tempo já realizado. O momento linear se conserva?
%\item O resultado obtido para a conservação (ou não) do momento linear observando-se o deslocamento do centro de massa do sistema está de acordo com o obtido analisando-se o movimento dos dois carrinhos separadamente?  
%\item Determine a velocidade do centro de massa antes e depois da colisão a partir do ajuste linear. Discuta.

%{\bf Observação:} Lembre de colocar todos os cálculos de propagação de incertezas num Apêndice, anexado ao Relatório, claramente explicados.