% --------------------------------------------------------------------
% --------------------------------------------------------------------
% Modelo de Trabalho Acadêmico utilizando classe repUERJ para elaboração
% de teses, dissertação e trabalhos monográficos em geral.
%
% Este arquivo está editado na codificação de caracteres UTF-8.
%
% As referencia estão baseadas no modelo bibtex e citação em autor-data
%
% Este modelo foi criado por Dr. Luís Fernando de Oliveira.
% Professor Adjunto do Departamento de Física Aplicada e Termodinâmica
% Instituto de Física Armando Dias Tavares
% Universidade do Estado do Rio de Janeiro - UERJ
%
% A classe repUERJ.cls foi criada a partir do código original
% disponibilizado pelo grupo CódigoLivre (coordenado por
% Gerald Weber).
% Foram feitas adequações para implementação das normas de elaboração
% de teses e dissertações da UERJ.
%
% Os estilos repUERJformat.sty codificam os elementos pré-textuais e
% pós-textuais.
% O estilo repUERJpseudocode.sty codifica a elaboração de algoritmos
% utilizando um glossário desenvolvido por mim (Luís Fernando), o mesmo
% usado em meu curso de Física Computacional.
%
% Todo este material está disponível também no meu site
%      http://sites.google.com/site/deoliveiralf
%
% As normas da UERJ para elaboração de teses e dissertações pode ser
% obtidas no documento disponível no site
%      http://www.bdtd.uerj.br/roteiro_uerj_web.pdf
%
% Agradecimentos ao grupo da Rede Sirius - BDTD e à Biblioteca Setorial
% da Física.
% --------------------------------------------------------------------
% --------------------------------------------------------------------
%
\documentclass[a4paper,12pt,oneside,onecolumn,final,fleqn]{repUERJ}
% ---
% Pacotes fundamentais
% ---
\usepackage[brazil]{babel}  % adequação para o português Brasil
\usepackage[utf8]{inputenc} % Determina a codificação utilizada
                            % (conversão automática dos acentos)
\usepackage{makeidx}        % Cria o índice
\usepackage{hyperref}       % Controla a formação do índice
\usepackage{indentfirst}    % Endenta o primeiro paragrafo de
                            % cada seção.
\usepackage{graphicx}       % Inclusão de gráficos
\usepackage{subfig}
\usepackage{amsmath}        % pacote matemático
% ---
% Pacote auxiliar para as normas da UERJ
% ---
\usepackage[frame=no,algline=yes,font=default]{repUERJformat}
\usepackage{repUERJpseudocode}

\usepackage[utf8]{inputenc}
\usepackage[T1]{fontenc}
\usepackage{amsmath, calc, xcolor}
% ---
% Pacotes de citacoes
% ---
%\usepackage[alf]{abntex2cite}

% ********************************************************************
% ********************************************************************
% Informações de autoria e institucionais
% ********************************************************************
% ********************************************************************

%---------------------------------------------------------------------
% Imagens pretextuais (precisam estar no mesmo diretório deste arquivo .tex)
%---------------------------------------------------------------------

\logo{logo_uerj_cinza.png}
\marcadagua{marcadagua_uerj_cinza.png}{1}{160}{255}

%---------------------------------------------------------------------
% Informações da instituição
%---------------------------------------------------------------------

\instituicao{Universidade do Estado do Rio de Janeiro}
            %{centro}
            {Centro de Tecnologia e Ci\^encias}
            {Instituto de F\'isica Armando Dias Tavares}

%---------------------------------------------------------------------
% Informações da autoria do documento
%---------------------------------------------------------------------

\autor{Mariana}
      {Soeiro}
      {M. S.}

\titulo{Estudo da Reconstrução e Identificação dos Fótons no Experimento ATLAS}
% se não for usar a quarta palavra chave, deixar o campo vazio: {}
\palavraschaves{ATLAS}
               {B\'oson de Higgs}
               {ALP}
               {}

\title{Estudo da Reconstrução e Identificação dos Fótons no Experimento ATLAS}
\keywords{first keyword}
         {second keyword}
         {third keyword}
         {}

\orientador{Prof. Dra.}
           {Marcia}{Begalli}
           {IF -- UERJ}

%coorientador é opcional
\coorientador{Prof. Dra.}
             {Yara}{ do Amaral Coutinho}
             {IF-- UFRJ}

%---------------------------------------------------------------------
% Grau pretendido (Doutor, Mestre, Bacharel, Licenciado) e Curso
%---------------------------------------------------------------------

\grau{Mestre} 
\curso{F\'isica}

% área de concentração é opcional
%\areadeconcentracao{área}

%---------------------------------------------------------------------
% Informações adicionais (local, data e paginas)
%---------------------------------------------------------------------

\local{Rio de Janeiro}
\data{06}{Agosto}{2020}

% ********************************************************************
% ********************************************************************
% Configurações de aparência do PDF final
% ********************************************************************
% ********************************************************************

% alterando o aspecto da cor azul
\definecolor{blue}{RGB}{41,5,195}
%\definecolor{apricot}{RGB}{251,206,177}

% informações do PDF
\hypersetup{
  %backref=true,
  %pagebackref=true,
  %bookmarks=true,
  unicode=false,
  pdftitle={\UERJtitulo},
  pdfauthor={\UERJautor},
  pdfsubject={\UERJpreambulo},
  pdfkeywords={PALAVRAS}{CHAVES}{\chaveA}{\chaveB}{\chaveC}{\chaveD},
  pdfproducer={\packagename},       % producer of the document
  pdfcreator={\UERJautor},
  colorlinks=true,       % false: boxed links; true: colored links
  linkcolor=black,       % color of internal links blue
  citecolor=black,       % color of links to bibliography blue
  filecolor=black,       % color of file links magenta
  urlcolor=black,
  bookmarksdepth=4
}